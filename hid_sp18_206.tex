% status: 0
% chapter: TBD

\title{IPL Cricket Match Predictor}


\author{Krish Hemant Mhatre}
\affiliation{%
  \institution{Indiana University}
  \city{Bloomington} 
  \state{Indiana} 
}
\email{kmhatre@iu.edu}



% The default list of authors is too long for headers}
\renewcommand{\shortauthors}{K. Mhatre}


\begin{abstract}
This paper describes the semester project built for ENGR-E-222.
\end{abstract}

\keywords{hid-sp18-206, cricket, IPL, Perceptron}


\maketitle

\section{Organization}

Krish Hemant Mhatre is sophomore at Indiana University from Mumbai, India. 
He is an Intelligent Systems Engineering major with concentration in Cyber Physical Systems. 
Krish is also planning to do a minor in Urban Planning and Community Development. 
He works as Engineering Tutor at School of Informatics, Computing and Engineering. 
He is interested in smart environments and robotics.
 

\section{Introduction}

The project is a cricket match predictor based on the data of IPL (seasons 2008-2017). The predictor used here is a simple perceptron (not to be mistaken as the Multi-Layered Perceptron).

The predictor is a RESTful service on Docker.
The dataset used is taken from kaggle.com provided by Mr. Manas Garg as Indian Premier League(Cricket).


\section{Code Analysis}
The code first downloads the data using requests. As it was unable for me to get the data directly from kaggle, I made a dropbox file for it. The program then partitions the data using three arguments - team1, team2 and test season. The data involving two teams is only written down and the data is divided between train and test according to the test season. 
After partitioning the data, it is simplified to binary to understand the pattern. The following characteristics - Home(1)-Away(0), Toss-Win(1)-TossLose(0) and Toss Decision - bat(1) and field(0) are used in training x array whereas previous results - Win(1) and Loss(0) - are used in training y array.
Then a perceptron (from sklearn) is used to predict the y array of the test season using the x array of the test season by studying the training data.
This is a type of simple neural network - Dataset - seasons - matches - each match with several attributes (home-away, toss-win, toss-decision, final result).



\section{Conclusion}

In the end, the program predicts a winner between the two teams and then compares the prediction of y array by perceptron with the actual results of the test season and hence gives the prediction accuracy. 


\begin{acks}

  The author would like to thank Dr. Geoffrey Fox and Dr. Gregor von Laszewski for their
  support and suggestions to write this paper.
  The author would also like to thank all the TAs of ENGR-E-222 for their constant guidance.

\end{acks}

\bibliographystyle{ACM-Reference-Format}
\bibliography{report} 

Garg, Manas. Kaggle - Indian Premier League (Cricket). matches.csv, www.kaggle.com/manasgarg/ipl/data.

Scikit Learn. http://scikit-learn.org/stable/modules/generated/sklearn.linearmodel.Perceptron.html

